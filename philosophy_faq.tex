\documentclass[oneside,final,14pt]{extreport}
\usepackage[utf8]{inputenc}
\usepackage[russianb]{babel}
\usepackage{vmargin}
\setpapersize{A4}
\setmarginsrb{2.5cm}{2cm}{1.5cm}{2cm}{0pt}{0mm}{0pt}{13mm}
\usepackage{indentfirst}
\sloppy
\begin{document}

\tableofcontents

\chapter{Введение}
\section{Что такое философия и зачем она нужна?}
Философия — особый вид человеческого знания. Сам термин “философия” греческого происхождения, он состоит из двух слов: “$\varphi \iota \lambda o$” — любовь и “$\sigma o \varphi \iota \alpha$” — мудрость. На русский язык это можно перевести как “любовь к мудрости” или, как в некоторых случаях говорилось в XIX, а иногда и в XX веке, как “любомудрие”.

Термин «философия» возникает в Древней Греции и традиционно связывается с именем Пифагора.

Предмет философии — это ответ на вопрос, чем в принципе занимается философия, какова направленность данного знания.

Предмет любого знания мы можем определять двумя способами. Мы можем либо подойти нормативно, то есть задать предмет, сформулировать ответ на этот вопрос нормативным образом, либо мы можем пойти историческим путем, то есть исходя из того, какую деятельность обычно называют подобным образом, применительно к чему употребляется данное понятие.

В случае с философией у нас возникают трудности и при первом, и при втором подходе. С одной стороны, на протяжении истории философии мы наблюдаем массу разных подходов, ни один из которых не является безусловно доминирующим. В связи с этим, хотя каждый из этих подходов отстаивает определенное понимание философии, настаивает на том, что философию надлежит понимать именно таким образом, ни один из этих подходов не становится монопольным, и поэтому нормативный подход всегда оказывается ограниченным.

Мы можем прибегнуть к историческому подходу и посмотреть, что на протяжении истории было принято называть философией. Однако если мы пойдём через историю, то увидим, что в каждую конкретную историческую эпоху под философией понимают весьма разные вещи. Мы увидим, что, например, древние греки — Фалес, Анаксимандр, Анаксимен, тот же самый Пифагор, уже помянутый, Платон или Аристотель — считают философией, например, то, что мы сейчас относим к физике, биологии, политической теории и массе других дисциплин.

Мы увидим, что в средние века под философией аналогично понимается самое разнообразное знание. Иными словами, мы попадаем в ситуацию, когда исторический подход демонстрирует нам, что в разные эпохи понимание философии совершенно различно.

Тогда, возможно, мы имеем дело с ситуацией семейного сходства, как это называл Людвиг Витгенштейн, а именно, когда мы смотрим на семейный портрет и видим, что на нем изображено одно семейство, все они похожи друг на друга, и вместе с тем если мы постараемся определить, обладают ли все члены этого семейства какой-либо одной чертой, связывающей их между собой, то мы со всей определенностью скажем, что такой одной черты нет ни для кого из них. Форма носа будет присуща восьмидесяти процентам, форма ушей присуща семидесяти процентам и так далее, и так далее.

Если мы, например, посмотрим на античную философию, то увидим, что предмет античной философии связан с предметом средневековой философии, а предмет средневековой связан с предметом философии Нового времени, который, в свою очередь, связан с Новейшим временем. Однако, если мы попытаемся сказать, что будет общего для всех эпох, для всех подходов в истории философии, то в рамках этой концепции окажется, что общего признака здесь не будет.

Однако и от подхода Витгенштейна мы вынуждены отказаться, поскольку если бы это было так, если бы действительно мы имели дело с ситуацией семейного сходства, то тогда прошлое философии нас интересовало бы исключительно с точки зрения истории науки, подобно тому как, когда мы занимаемся физикой, мы можем одновременно интересоваться и историей физики. Но наш ключевой интерес заключается в современной физической теории, в то время как в философии мы  постоянно обращаемся к прошлому.

Поэтому если отвечать совсем кратко, то предмет философии — это о главном, как говорил Плотин, это универсальное знание или, как писал Хосе Ортега-и-Гассет, философ — это специалист, но специалист особого рода. Это специалист, который занимается универсальным, в отличие от любого, например, конкретного научного знания, которое сосредоточено на какой-то особой предметной области. Это знание о том, как мы можем помыслить целое.

Отсюда вытекают и основные функции философии. Ключевая функция — мировоззренческая. Другая функция, о которой необходимо упомянуть, — методологическая, то есть способы нашего познания. И, наконец, третья функция — критическая, то есть способность критически относиться к окружающей действительности.

\section{Чем философия отличается от науки, религии и искусства?}


Философия является одной из форм мировоззрения — наряду с религиозным, мифологическим, научным — и вместе с тем особой формой знания. Тем самым необходимо отметить, что философия — это философия, философия не является чем-то другим. Философия — это не религия, не наука, не искусство.

Вместе с тем очень часто можно встретить утверждение, что философия является наукой. Здесь необходимо понимать, что сам термин "наука" употребляется в разных значениях.

Если под словом "наука" мы понимаем всякое рациональное и систематическое знание, то, безусловно, философия является наукой. Но точно так же наукой, например, будет теология. Однако если мы понимаем под наукой, ну, например, вслед за Карлом Раймундом Поппером знание, которое является верифицируемым и фальсифицируемым, то есть, другими словами, содержит в себе указание на критерий собственного опровержения, то есть может быть принципиально опровергнуто, то в этом смысле философия не является научным знанием.

Попросту говоря, мы всегда должны держать в уме, что у слова "наука", как и у слова "философия", есть различные значения. И, во всяком случае, философия точно не является наукой в том же смысле, как когда мы говорим, например, о том, что наукой является физика или химия. А то, что философия является особым видом знания, мало что проясняет, поскольку нам необходимо конкретизировать, в чем именно эта самая особенность, где пролегает различие между философией и другими видами знания.

Легче всего пойти по пути выявления сходств и различий. Так, применительно к науке, мы можем сказать, что философия роднится с наукой тем, что она является рациональным знанием. Во-вторых, она является систематическим знанием. Попросту говоря, перед нами есть некая система понимания, где не просто отдельные факты, не просто отдельная информация, а возможность получения нового знания на основе предшествующего за счет построения системных связей. 

В-третьих, подобно науке, философия является понятийной формой знания. Она работает посредством понятий, их разработки, их уточнения и так далее. Однако, в отличие от науки, философия работает с предельными понятиями. В отличие от остальных видов понятий, предельные понятия не имеют понятий более высокого уровня общности. Следовательно, здесь мы встречаемся с логическим парадоксом, а именно, если вспомнить, как определяется само по себе понятие. Понятие — это то, что определяется, то есть чему полагается предел, то есть мы определяем понятие через нечто иное. И, как писал Аристотель, древнегреческий философ четвертого века, один из создателей логики, определить понятие — значит указать на родовую принадлежность данного понятия и на видовое отличие.

В случае с философией и теми понятиями, которыми оперирует философия, у философских понятий нет родового, родовой принадлежности, нет более высоких понятий. Следовательно, получается, что философия вынуждена использовать другие формы не в силу недостаточности по сравнению с наукой, а в силу специфики предмета своего мышления.

В отличие от религии, философия является рациональной и доказательной формой знания, то есть это способ осмысления в понятиях, который предполагает изначальную критичность. Иными словами, это не означает, что в философии нет чего-либо, что мы принимаем на веру, как в религии. Безусловно, мы всегда должны откуда-то начинать. Ключевое заключается в том, что в философии любой тезис, который мы принимаем и которым мы оперируем, может быть принципиально поставлен под сомнение.

И, наконец, в отличие от искусства, философия работает с понятиями, а не с образами. А это одновременно и преимущество философского знания, поскольку понятийная форма придает философскому знанию универсальность и сообщаемость, в отличие от искусства, а с другой стороны, это же означает, что, поскольку мы мыслим в образах, мы способны воспринимать непосредственно, мы способны воспринимать больше, чем нами осознается. Философия всегда предполагает, осознание — это рефлексия. И, следовательно, получается, что мы не можем обладать философским знанием и не можем разделять философские концепции, не зная об этом. Именно поэтому, когда мы говорим, что у каждого человека есть своя философия, то мы имеем в виду, что убеждение или представление каждого человека могут быть представлены в философской форме, но отнюдь не в том смысле, что он обладает подобным знанием.

\chapter{Мир и познание}
\section{Что такое мир вообще?}


Когда мы пытаемся осмыслить мир вообще, мир как целое, то перед нами возникает вопрос: как мы можем собрать всю реальность, все многообразие всего сущего в некий порядок, посредством чего мы можем это объяснить? Соответственно, в рамках европейской философии Нового времени, отсылающая здесь и к античной и к средневековой традиции, мы логически можем представить себе несколько вариантов.

Во-первых, мы можем предположить, что мир имеет разные основания, множественность начал. И тогда мы говорим о том, что реальность сама по себе плюральна – перед нами есть некая множественность оснований производная. Однако остается вопрос: каким образом тогда эта множественность тем не менее собирается в некое единство?

Другой подход, который можно себе представить, — это дуалистический подход. В традиционной истории философии пример дуализма служит картезианство. Однако стоит отметить, что, вообще, дуалистическое рассмотрение мира в целом наталкивается на неизбежное ограничение. Более того, как раз в картезианстве мы видим, как эта логика не срабатывает, поскольку в любом случае вся реальность у самого Декарта оказывается связана с Богом, который мыслится как единственная субстанция. И уже по отношению к нему мы говорим о субстанции мыслящей, о субстанции протяженной, то есть выводим порядок духовного (порядок разумного) и порядок материального. Тем самым, раз мы даже предполагаем наличие множественности или двойственности начал, эти начала должны быть включены в некое единство.

Отсюда вытекает обоснованность монистического взгляда, то есть представления о том, что реальность едина, а раз она едина, то она должна быть помыслена нами через некий один принцип, одно начало. Либо мы вынуждены сказать, что этот принцип, это начало нами не мыслимо. Но из этого вытекает ограничение наших познавательных способностей, но не тезис о множественности миров, о множественности самих субстанциальных оснований.

Итак, если мы исходим из монистического представления, то есть представления о том, что реальность фундаментально едина, то, в рамках опять же европейской традиции, начиная с противопоставления Платона и Демокрита, мы можем говорить о двух подходах, об идеалистическом и материалистическом описании. А именно — что является первичным: некое материальное начало, производным от которого является идеальное, либо об идеальном, тогда материальное мыслится как некое инобытие, иное по отношению к идеальному, порождение этого идеального.

Данные подходы принципиально важны, поскольку они влекут за собой разные принципы объяснения, описания реальности. А именно, если мы занимаем материалистическую позицию, то объяснить, понять что-либо означает свести феномены, нами наблюдаемые, самые различные — говорим ли мы о феноменах интеллектуальной жизни, духовной жизни — свести все это к неким материальным основаниям. Если нам не удается осуществить данное сведение, то это проблема ограниченности наших познавательных способностей в данный момент, отсутствие в данный момент подходящих теорий. Однако принципиально с точки зрения материалистического подхода — все существующее через ту или иную степень опосредования восходит к материальному основанию.

Напротив, если мы придерживаемся идеалистического взгляда, то объяснить что-либо означает понять идею, понять исходный принцип этого существования, и тем самым материальное оказывается только закреплением в реальности другой формы, другим способом существования того самого смысла. Тогда как именно сфера смыслов, осмысленное является подлинной реальностью.

\end{document}

